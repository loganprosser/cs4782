\documentclass{article}
\usepackage{amsmath, amssymb, amsfonts}
\usepackage{fullpage}
\usepackage{enumerate}
\usepackage[linesnumbered,ruled,vlined]{algorithm2e}
\usepackage[usenames,dvipsnames]{xcolor}
\usepackage[colorlinks,allcolors=RoyalBlue]{hyperref}
\usepackage{enumitem}
\usepackage{graphicx} % Required for inserting images
\usepackage[margin=1in]{geometry}
\usepackage{xcolor}
\usepackage{amsmath}
\usepackage{tikz}
\usepackage{tkz-base}
\usepackage{tkz-euclide}
\usepackage{comment}

\usepackage[T1]{fontenc}
% \usepackage{palatino}
\usepackage{libertine, libertinust1math}
\usepackage{titlesec}
% \usepackage{newtxtext,newtxmath}
% \usepackage{kpfonts}

\renewcommand{\familydefault}{\sfdefault}
\renewcommand{\sfdefault}{LinuxLibertineT-TLF}


% For environments and alias
\usepackage{amsthm}
\usepackage{mdframed}
\usepackage{mathtools}
\usepackage{cleveref}
\newmdtheoremenv{assumption}{Assumption}

\DeclareMathOperator*{\argmin}{arg\,min}
\DeclareMathOperator*{\argmax}{arg\,max}
\newcommand{\loss}[1]{\mathcal{L}_\text{#1}}

\newcommand{\R}{\mathbb{R}}
\newcommand{\C}{\mathbb{C}}
\newcommand{\Z}{\mathbb{Z}}
\newcommand{\N}{\mathbb{N}}
\newcommand{\F}{\mathbb{F}}
\newcommand{\E}{\mathbb{E}}
\newcommand{\cO}{\mathcal{O}}
\newcommand{\ctO}{\Tilde{\cO}}
\newcommand{\mdot}{\:\cdot\:}
\newcommand{\jacf}{\nabla f}
\newcommand{\hesf}{\nabla^2 f}

\newcommand{\x}{\mathbf{x}}
\newcommand{\y}{\mathbf{y}}
\newcommand{\w}{\mathbf{w}}

\DeclarePairedDelimiter{\norm}{\lVert}{\rVert}
\DeclarePairedDelimiterX{\inner}[2]{\langle}{\rangle}{#1,#2}
\DeclarePairedDelimiter{\Set}{\lbrace}{\rbrace}
\DeclarePairedDelimiter{\abs}{|}{|}
\DeclarePairedDelimiter{\ceil}{\lceil}{\rceil}
\DeclarePairedDelimiter{\floor}{\lfloor}{\rfloor}
\DeclarePairedDelimiter{\paren}{(}{)}
\DeclarePairedDelimiter{\bracket}{[}{]}
\DeclarePairedDelimiter{\curly}{\lbrace}{\rbrace}
\DeclarePairedDelimiter{\card}{|}{|}
\DeclarePairedDelimiter{\norms}{\lVert}{\rVert^2}
\DeclarePairedDelimiter{\fnorm}{\lVert}{\rVert_F}
\DeclarePairedDelimiter{\infnorm}{\lVert}{\rVert_\infty}

\newcommand{\mincurly}[1]{\min\curly*{#1}}
\newcommand{\maxcurly}[1]{\max\curly*{#1}}
\newcommand{\Exv}[1]{\E\bracket*{#1}}
\newcommand{\bigO}[1]{\cO\paren*{#1}}
\newcommand{\bigtO}[1]{\ctO\paren*{#1}}
\newcommand{\prob}[1]{P\paren*{#1}}
\newcommand{\condprob}[2]{P\paren*{#1\mid#2}}

\newenvironment{solution}{
    \color{Red}
    \textbf{Solution:}
}

\title{CS 4782 Coding Assignment 2 Written Responses\vspace{-10pt}}
\author{Due: 3/6/25 11:59 PM on Gradescope}
\date{Late submissions accepted until 3/8/25 11:59 PM}

\begin{document}
    \maketitle
    \textbf{Note: For homework, you can work in teams of up to 2 people. Please include your teammates’ NetIDs and names on the front page and form a group on Gradescope. (Please answer all questions within each paragraph.) Please show all the relevant steps in your solutions. }
\maketitle
\section*{Question 1:}

Based on the provided implementation of \textit{PositionalEncoding}, what is the formula used to assign an embedding to each position/index $i$ in the input sequence? What is one benefit of using this function/formula specifically to generate position embeddings? 


\section*{Question 2:}

Describe the matrix size transformations as they happen from steps 1-5. Specificaly, what is the shape of: 
\begin{enumerate}
    \item the input to the \textbf{split heads} function? 
    \item the output from the \textbf{split heads} function?
    \item \textbf{multi-head} variable in the description above?
    \item final output of \textbf{Multi-Head(Q, K, V)}?
\end{enumerate}

Answers expected as matrix shapes in terms of $n$, $d_{model}$, $d_q$, $d_k$, $d_v$ and $h$. You can use the additional variable $b$ to represent batch size.


\section*{Question 3:}

What do you notice about the graph above and the differences between the five models you have implemented? Are the results consistent with what you expected? How do pre-trained models compare to models you trained from scratch? Write 3-4 sentences below.

\end{document}